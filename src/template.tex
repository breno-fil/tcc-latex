\documentclass[a4paper, 12pt]{article}
\usepackage[a4paper, rmargin=3cm, lmargin=3cm, top=2.5cm, bottom=2.5cm]{geometry}
\usepackage{mathtools}
\usepackage{bookmark}
\usepackage{subcaption}
\usepackage{amsthm,amsmath,amssymb}
\usepackage{enumitem}
\usepackage[ruled, noline]{algorithm2e}
\usepackage{fontspec}

\usepackage{setspace}
\setstretch{1.5}
\setlength{\parindent}{1.25cm}
\setlength{\parskip}{6px}

\usepackage{unicode-math}
\setmathfont{Stix Two Math}
\setmathfont{TeX Gyre Pagella Math}[
   range=bb,
   Scale=MatchUppercase,
   version=pagella
]

\usepackage{polyglossia}
\setdefaultlanguage{portuguese} % might be portuguese, though referencing may present incompatibilities
\setotherlanguages{english} % secondary language for abstract and stuff
\usepackage{times} % not recommended, but it is the standard package for Times included in LuaTeX and ready to go for Overleaf, see alternative option below (which require local installation of LaTeX)

% in case you have a local installation for the font Times, this is preferrable, especially if you have Small Caps support
%\setmainfont{times}[
%  % In my case, needs to point for MS Office location since regular MacOS Times doesn't support small caps
%  Path           = /Applications/Microsoft Word.app/Contents/Resources/DFonts/,
%  Extension      = .ttf ,
%  BoldFont       = *bd ,
%  ItalicFont     = *i ,
%  BoldItalicFont = *bi,
%  Ligatures      = Rare, %for some unknown reason, Times put Common ligatures under Rare
%]

% or simply, in case you have the legal font Times LT Std installed in your local machine
%\setmainfont{Times LT Std}[Ligatures = Common] % name might differ

% useful command defition for writting mathematics: QED, norm, set cardinality and equals by definition
\renewcommand{\qedsymbol}{$\blacksquare$}
\newcommand{\norm}[1]{\left\lVert#1\right\rVert}
\newcommand{\card}[1]{\lvert#1\rvert}
\newcommand*{\defeq}{\mathrel{\vcenter{\baselineskip0.5ex \lineskiplimit0pt
                     \hbox{\scriptsize.}\hbox{\scriptsize.}}}%
                     =}

% useful command definition for SCB bracket citation
\newcommand{\citeb}[1]{\bibleftbracket\cite{#1}\bibrightbracket}

% math numbering scheme, speak with your supervisor in case you want to change it
\newtheorem{theorem}{Theorem}[section]
\newtheorem*{definition}{Definition}
\newtheorem{corollary}{Corollary}[theorem]
\newtheorem{lemma}[theorem]{Lemma}


% reference style (Sociedade Brasileira de Computação is compliant with Chicago author-date)
% as per https://presencial.unifcv.edu.br/arquivos/orientacao_para_artigos_area_informatica.pdf
\usepackage[authordate, strict, backend=biber, autolang=other]{biblatex-chicago}
\addbibresource{references.bib}
\DeclareDelimFormat{nameyeardelim}{\addcomma\space}
\setlength\bibitemsep{6.0pt}

% flush all sections right
\usepackage{sectsty}
\sectionfont{\raggedleft}

% used only for text samples, can be safely removed in the final document
\usepackage{lipsum} 
\usepackage{metalogo}

\begin{document}
\pagenumbering{roman}
\setcounter{page}{1}
\thispagestyle{empty}
    \begin{center}
        \includegraphics[scale=0.18]{images/unirio.png}\\
        \fontsize{13}{15}
        \textsc{
        Universidade Federal do Estado do Rio de Janeiro\\
        Centro de Ciências Exatas e Tecnológicas\\
        Escola de Informática Aplicada\\}
        \vspace{2.8cm}
        Work's title\\
            
        \vspace{2.8cm}
        Author's name
        \vspace*{\fill}\\
        
        \textsc{Rio de Janeiro, RJ -- Brasil\\
        (Mês), (Ano)}
    \end{center}
    \clearpage

    \begin{center}
        Work's title
        \vskip 0.5cm
        Author's name
        \vskip 2.0cm
    \end{center}
    \begin{flushright}
        \parbox{8.0cm}{
        Projeto de graduação apresentado à Escola de Informática Aplicada
        da Universidade Federal do Estado do Rio de Janeiro (UNIRIO) como
        cumprimento de requerimento parcial para obtenção título de Bacharel em
        Sistemas de Informação.}
        \vskip 1.5cm
        Approved by:
        \vskip 1.5cm
        \rule{10.0cm}{.1mm}

        Supervisor, D.Sc. -- UNIRIO
        \vskip 1.0cm

        \rule{10.0cm}{.1mm}

        Supervisor 2, D.Sc. -- UNIRIO
        \vskip 1.0cm

        \rule{10.0cm}{.1mm}

        Supervisor 3, D.Sc. -- XXXX
        \vskip 1.0cm
    \end{flushright}
    \vspace{\stretch{1}}
    \begin{center}
        \textsc{Rio de Janeiro, RJ -- Brasil} \\ \textsc{(Mês), (Ano)}
    \end{center}
    % Fim da folha de rosto

    \clearpage
    \begin{flushright}
        \section*{Acknowledgments}    
    \end{flushright}
    \lipsum[1]
    \clearpage
    \begin{abstract}
        \lipsum[1]
        
        {\bf Palavras-chave:} Separadas por vírgulas.
    \end{abstract}
    \clearpage
    \begin{english}
        \begin{abstract}
            \lipsum[1]

            {\bf Keywords:} Between commas.
        \end{abstract}
    \end{english}
    \clearpage
    \tableofcontents
    \clearpage
    \listoffigures
    \clearpage
    \listoftables
    \clearpage
    \listofalgorithms
    % Não sei como fazer para aparecer em português. Dê seus pulos.
    \clearpage
    \pagenumbering{arabic}
    \setcounter{page}{1}
    \section{Introdução}

    \subsection{Motivação}
    \lipsum[1-2]
    \subsection{Objetivos}
    \lipsum[3-4]
    \subsection{Organização do texto}
    \lipsum[5-6]
    \clearpage
    \section{Configurações}
    \subsection{Ligaduras}
    Se a versão da fonte Times New Roman (ou Times) tiver sido escolhida adequadamente
    (ao menos versão 5.01.3x), as ligaduras entre as letras deve ser exibida
    corretamente e as seguintes comparações estão corretas: \\
    
    \begin{center}
        \Huge
        f\/f\/i \(\neq\) ffi \\
        f\/i \(\neq\) fi \\
        f\/l \(\neq\) fl \\
        f\/f\/l \(\neq\) ffl \\
    \end{center}

    \subsection{Engine}
    Este documento foi pensado para ser compilado em Lua\TeX \,e Biber, não sendo
    garantido seu funcionamento com \LaTeXe \,ou \XeLaTeX.

    \clearpage
    \section{Conclusão}
    \clearpage

    \section*{Referências}
    Instruções de bibliografia a seguir foram retiradas do manual de referência
    da Sociedade Brasileira de Computação \citeb{sbc}:
    % no meu TCC, eu utlizei somente \citep. Fale com seu supervisor.

    As referências bibliográficas devem ser de entendimento único e uniformes.
    Nós recomendamos dar ao autor nomes de referências em colchete, e.g.
    \citeb{knuth}, \citeb{smith};
    ou datas nos parênteses, \textcite{knuth}, \textcite{smith}.
    
    As referências devem ser listadas usando o tamanho de fonte de 12 pontos,
    com 6 pontos do espaço antes de cada referência.
    A primeira linha de cada referência não deve ser recuada,
    quando a subseqüente dever ser recuada 0.5 cm.
    \clearpage
    \nocite{*}
    % flush section font left
    \sectionfont{\raggedright}
    \printbibliography
\end{document}
